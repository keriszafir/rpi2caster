\documentclass[draft, 12pt, a4paper]{book}
\title{rpi2caster}
\begin{document}
\chapter{Installation}
\section{Prerequisites}
To operate the device, you need to have available at your casting workshop:
\begin{enumerate}
\item 24V DC @ 2A power supply, central or individual
\item 1bar / 15psi air connection 6mm calibrated plastic tube (connected to the composition caster's air supply or a pressure regulator), air must be dry and clean 
\item Ethernet connection to the local area netwotk is recommended
\item optional peripherals like a USB wireless LAN interface, flash drive, 
\item optionally, on select models: HDMI video out to external display and USB keyboard/mouse,
\item one or two paper gaskets - strips of \textsc{Monotype} ribbon, 2" - 5cm long, with a row of holes in all 31 channels,
\item a bit of vaseline or \textsc{Monotype mould oil}.
\end{enumerate}
\section{Installing the attachment on the caster}
To connect the device, you need to:
\begin{enumerate}
\item Slide the \textsc{winding spool} into the clamp so that its flange has ca. 1/4" - 6mm clearance from the \textsc{computer control attachment's} enclosure, and fasten the screws.
\item Attach the assembly to the caster's \textsc{paper tower} the same way the original \textsc{winding spool} is attached.
\item Put the air coupling block clamp on the outer edges of the \textsc{pin wheels} on the caster's \textsc{paper tower} and lift it.
\item Spread some vaseline grease or \textsc{Monotype mould oil} on the paper gaskets in a thin uniform layer.
\item Put the gaskets on the \textsc{cross girt} so that the holes are aligned.
\item Slide the air coupling block under the clamp and position it so that the holes in the block and the \textsc{paper tower cross girt} are aligned. Disconnect the leftmost and rightmost air tubes and blow the compressed air into the connections to check if the air is directed into \textsc{air pins} M and 14.
\item Fasten the screws in the clamp by hand so that the air coupling block is pressed firmly against the \textsc{cross girt}.
\item Connect the \textsc{connecting hook} with the round stud on the machine cycle sensor activation lever.
\item Connect the air and power supplies, and preferably the local area network connection, and power on the \textsc{computer control attachment}.
\item Turn the machine by hand and adjust the rotating disc at the end of the machine cycle sensor activation lever so that the \textit{Cycle} LED is lit when the \textsc{clamping rod} is in the bottom position (i.e. when the air is passed through the ribbon in the traditional casting procedure).
\end{enumerate}
\chapter{Connection}
\section{Finding your device on the local area network}

\end{document}